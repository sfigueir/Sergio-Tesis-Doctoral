%!TEX root = ../main.tex


    \color{magenta}
    Propuesta: acá poner la gramática y semántica de la lógica proposicional

    Todo lo que está en magenta en los caps 4 y 5 iría en esta parte 2

    Eg:
    
    Fig \ref{PCFG}, solo la parte de \grambool pero con una cantidad no acotad de proposiciones em ve de fijarlas en 4 . No incluir \gramboolxor acá.

    Definir formalmente $\sem{\varphi}$

    Definir el tamaño de la formula. Usar notación común para cap 4 y 5

    definir  \mdlbool para complejidad relativa a \grambool, pero dejar la definición de
      \mdlboolxor para complejidad relativa a \gramboolxor adentro del cap 4 porque solo aparece ahí. 

\color{black}

----------------------------------------

\paragraph{La gramática \grambool.}
Para un conjunto fijo $\propvars=\{p_1,\dots,p_k\}$ de variables proposicionales, definimos la gramática de la lógica proposicional $\grambool$ en la Figura \ref{fig:grambool}.
%
\begin{figure}
\begin{eqnarray*}
\start &\to&\bool\\
\bool &\to&(\bool \oand \bool) \\
\bool &\to&(\bool \oor \bool) \\
\bool &\to&\atom\\
\atom &\to& p_i \\
\atom &\to&\lnot p_i 
\end{eqnarray*}
donde $i\in\{1,\dots,k\}$.
\caption{La gramática \grambool para la lógica proposicional.}
\end{figure}\label{fig:grambool}

Las {\em fórmulas} de \grambool serán las palabras del lenguaje definido por \grambool. Usaremos indistintamente \grambool para referirnos a la gramática o al lenguaje que representa (quedando siempre claro del contexto cuál de las dos aplica).


\paragraph{La semántica de \grambool.}
Una {\em valuación} es una función $v:\propvars\to\{0,1\}$. Equivalentemente, una valuación puede ser vista como un vector $v$ en $\{0,1\}^k$, donde notamos con $v(i)$ a la $i$-ésima coordenada de $v$. Informalmente, una valuación asigna valores de verdad a las variables proposicionales de $\propvars$ (la $i$-ésima coordenada de una valuación $v$ es $1$ si y solo si la variable proposicional $p_i$ es verdadera según $v$) y permite derivar el valor de verdad de fórmulas de $\grambool$.  

La {\em semántica} de una fórmula $\varphi$ será un conjunto (finito) de valuaciones sobre $\propvars$. Intuitivamente, este conjunto será su {\em extensión}, es decir, el conjunto de valuaciones (sobre el conjunto \propvars) que hacen verdadera a $\varphi$. 
%
Formalmente, la semántica de una fórmula $\varphi\in\grambool$ (para el conjunto $\propvars$ fijo) se nota $\sem{\varphi}$ y se define recursivamente del siguiente modo:
%
\begin{eqnarray*}
\sem{p_i} &=& \{ v\in\{0,1\}^k \mid v(i)=1\}\\
\sem{\lnot p_i} &=& \{ v\in\{0,1\}^k \mid v(i)=0\}\\
\sem{\varphi\wedge\psi} &=& \sem{\varphi}\cap\sem{\varphi}\\
\sem{\varphi\vee\psi} &=& \sem{\varphi}\cup\sem{\varphi}
\end{eqnarray*}
Esta definición formaliza la idea intuitiva de los operadores de conjunción ($\wedge$), y disyunción ($\vee$). Observar que en \grambool la negación solo aparece en frente de una variable proposicional. Por ejemplo, si $k=3$ (es decir, $\propvars=\{p_1,p_2,p_3\}$) entonces $\sem{p_2}=\{(0,1,0), (1,1,0), (0,1,1), (1,1,1)\}$,  $\sem{p_3}=\{(0,0,1), (0,1,1), (1,0,1), (1,1,1)\}$, y $\sem{p_2\wedge p_3}=\{(0,1,1), (1,1,1)\}$.

Llamamos {\em concepto} a un conjunto de valuaciones sobre la gramática \grambool. Puesto que el conjunto de conectivos de \grambool es adecuado, cualquier concepto $\con$ podrá ser expresado por una fórmula $\varphi\in\grambool$, en el sentido de que $\sem{\varphi}=\con$. De hecho, para cada concepto $\con$ existirán infinitas fórmulas que lo expresan.

\paragraph{Complejidad para \grambool.} Definimos $|\varphi|$, el {\em tamaño} de la fórmula $\varphi\in\grambool$, del siguiente modo:
\begin{eqnarray*}
|p_i| &\eqdef& 1\\
|\lnot p_i| &\eqdef& 1\\
|\varphi*\psi| &\eqdef& 1+|\varphi|+|\psi|
\end{eqnarray*}
donde $*\in\{\wedge,\vee\}$.

Si $\con$ es un conjunto de valuaciones sobre $\propvars$, definimos la {\em $\grambool$-complejidad de la descripción mínima} de $\con$, notada $\mdlbool(\con)$, como el tamaño de la fórmula más corta cuya semántica es $\con$, es decir,
$$
\mdlbool(\con)\eqdef\min\{|\varphi|\mid \varphi\in\grambool,\sem{\varphi}=\con\}.
$$
Por ejemplo, se puede ver que si $\con=\{(0,1,1), (1,1,1)\}$, entonces $\mdlbool(\con)=3$, que corresponde a la fórmula minimal $p_2\wedge p_3$, de tamaño 3. Observar que la complejidad de un concepto se refiere al tamaño de alguna de las fórmulas minimales cuya semántica coincide con dicho concepto, pero que podría haber más de una fórmula minimal.


    \color{black}