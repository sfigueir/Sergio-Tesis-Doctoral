%!TEX root = ../main.tex

\rhead{}
\lhead{}
\renewcommand{\headrulewidth}{0pt}
\begin{center}
    \section*{Resumen}
\end{center}

En las últimas dos décadas, distintas técnicas de ingeniería reversa del aprendizaje en humanos han influenciado el éxito de los algoritmos de aprendizaje automático. Técnicas como el aprendizaje profundo han alcanzado resultados notables en varios dominios como la detección visual de objetos, el reconocimiento de voz, o las traducciones automáticas, entre otros. Sin embargo, pese a que aprender a partir de pocos datos es una capacidad cotidiana de la mente humana, las técnicas actuales de aprendizaje automático no dan las mismas garantías al respecto.

\medskip

Investigaciones previas sobre modelos computacionales de la cognición humana han propuesto la idea de que la habilidad ubicua del ser humano para hacer predicciones sobre cuerpos ralos de datos se basa en el uso de modelos probabilísticos donde el conocimiento se representa en espacios adecuadamente estructurados sobre los que se aplican reglas de inferencia. Estos trabajos están revalorizando la hipótesis de Jerry Fodor que explica al pensamiento humano en una suerte de lenguaje mental llamado {\em Lenguaje del Pensamiento} compuesto por un conjunto de símbolos atómicos que pueden ser combinados en estructuras más complejas a partir de reglas combinatorias.

\medskip

En este trabajo diseñamos y evaluamos distintos modelos del Lenguaje del Pensamiento para explicar el aprendizaje humano con pocos datos en diversos dominios: secuencias binarias en el dominio visual y auditivo, secuencias geométricas en el campo visual, y conceptos lógicos. En nuestros modelos suponemos que el Lenguaje del Pensamiento actúa como un lenguaje de programación capaz de generar programas para modelar conceptos del mundo, y explicamos el aprendizaje como un proceso de inferencia probabilística sobre estos programas o con un enfoque de longitud mínima de descripción basado en las nociones de complejidad algorítmica. Proponemos, a su vez, distintas técnicas para mejorar el proceso de construcción y validación de los modelos del Lenguaje del Pensamiento con el objetivo de hacerlos más dinámicos y robustos.
\vfill


\noindent\textbf{Palabras Clave:} Lenguaje del Pensamiento, Inferencia Bayesiana, Complejidad de Kolmogorov, Longitud Mínima de Descripción.