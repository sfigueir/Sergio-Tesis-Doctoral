1.1.2 Thinking humanly: The cognitive modeling approach

For example, Allen Newell and Herbert Simon, who developed GPS, the “General Problem Solver” (Newell and Simon 1961), were not content merely to have their program solve problems correctly. They were more concerned with comparing the sequence and timing of its reasoning steps to those of human subjects solving the same problems. The interdisciplinary field of cognitive science brings together computer models from AI and experimental techniques from psychology to construct precise and testable theories of the human mind.

Cognitive science is a fascinating field in itself, worthy of several textbooks and at least one encyclopedia (Wilson and Keil 1999). We will occasionally comment on similarities or differences between AI techniques and human cognition. Real cognitive science, however, is necessarily based on experimental investigation of actual humans or animals. We will leave that for other books, as we assume the reader has only a computer for experimentation

1.2.8 Linguistics

In 1957, B. F. Skinner published Verbal Behavior. This was a comprehensive, detailed account of the behaviorist approach to language learning, written by the foremost expert in the field. But curiously, a review of the book became as well known as the book itself, and served to almost kill off interest in behaviorism. The author of the review was the linguist Noam Chomsky, who had just published a book on his own theory, Syntactic Structures. Chomsky pointed out that the behaviorist theory did not address the notion of creativity in language—it did not explain how children could understand and make up sentences that they had never heard before. Chomsky’s theory—based on syntactic models going back to the Indian linguist Panini (c. 350 BCE)—could explain this, and unlike previous theories, it was formal enough that it could in principle be programmed.

Modern linguistics and AI, then, were “born” at about the same time, and grew up together, intersecting in a hybrid field called computational linguistics or natural language processing. The problem of understanding language turned out to be considerably more complex than it seemed in 1957. Understanding language requires an understanding of the subject matter and context, not just an understanding of the structure of sentences. This might seem obvious, but it was not widely appreciated until the 1960s. Much of the early work in knowledge representation (the study of how to put knowledge into a form that a computer can reason with) was tied to language and informed by research in linguistics, which was connected in turn to decades of work on the philosophical analysis of language.


1.3.5 The return of neural networks (1986–present)

These so-called connectionist models were seen by some as direct competitors both to the symbolic models promoted by Newell and Simon and to the logicist approach of McCarthy and others. It might seem obvious that at some level humans manipulate symbols—in fact, the anthropologist Terrence Deacon’s book The Symbolic Species (1997) suggests that this is the defining characteristic of humans. Against this, Geoff Hinton, a leading figure in the resurgence of neural networks in the 1980s and 2010s, has described symbols as the “luminiferous aether of AI”—a reference to the non-existent medium through which many 19th-century physicists believed that electromagnetic waves propagated. Certainly, many concepts that we name in language fail, on closer inspection, to have the kind of logically defined necessary and sufficient conditions that early AI researchers hoped to capture in axiomatic form. It may be that connectionist models form internal concepts in a more fluid and imprecise way that is better suited to the messiness of the real world. They also have the capability to learn from examples—they can compare their predicted output value to the
true value on a problem and modify their parameters to decrease the difference, making them more likely to perform well on future examples.


1.3.6 Probabilistic reasoning and machine learning (1987– present)
The brittleness of expert systems led to a new, more scientific approach incorporating probability rather than Boolean logic, machine learning rather than hand-coding, and experimental results rather than philosophical claims.14 It became more common to build on existing theories than to propose brand-new ones, to base claims on rigorous theorems or solid experimental methodology (Cohen, 1995) rather than on intuition, and to show relevance to real-world applications rather than toy examples.

1988 was an important year for the connection between AI and other fields, including statistics, operations research, decision theory, and control theory. Judea Pearl’s (1988) Probabilistic Reasoning in Intelligent Systems led to a new acceptance of probability and decision theory in AI. Pearl’s development of Bayesian networks yielded a rigorous and efficient formalism for representing uncertain knowledge as well as practical algorithms for probabilistic reasoning. Chapters 12 to 16 cover this area, in addition to more recent developments that have greatly increased the expressive power of probabilistic formalisms; Chapter 20 describes methods for learning Bayesian networks and related models from data.

=== Summary

Philosophers (going back to 400 BCE) made AI conceivable by suggesting that the mind is in some ways like a machine, that it operates on knowledge encoded in some internal language, and that thought can be used to choose what actions to take.

Mathematicians provided the tools to manipulate statements of logical certainty as well as uncertain, probabilistic statements. They also set the groundwork for understanding computation and reasoning about algorithms.