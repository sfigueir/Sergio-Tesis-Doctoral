%!TEX root = ../main.tex

\thispagestyle{empty}

\ 

\vspace{3cm}

\begin{flushright}
\textit{A Alejandra y los sueños colectivos}
\end{flushright}

\newpage
\thispagestyle{empty}

\ 

\vspace{3cm}

\begin{flushright}
\textit{No sólo le costaba comprender que el símbolo genérico {\em perro} abarcara tantos individuos dispares de diversos tamaños y diversa forma; le molestaba que el perro de las tres y catorce (visto de perfil) tuviera el mismo nombre que el perro de las tres y cuarto (visto de frente) (...) Había aprendido sin esfuerzo el inglés, el francés, el portugués, el latín. Sospecho, sin embargo, que no era muy capaz de pensar. Pensar es olvidar diferencias, es generalizar, abstraer. En el abarrotado mundo de Funes no había sino detalles, casi inmediatos.
\\
\medskip
{\em J.L. Borges. Funes, el memorioso.~\cite{funes}}
}
\end{flushright}
\widesanti{revisar la cita. 1935-1944?}