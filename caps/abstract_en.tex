%!TEX root = ../main.tex

\rhead{}
\lhead{}
\renewcommand{\headrulewidth}{0pt}
\begin{center}
    \section*{Abstract}
\end{center}

In the last two decades, different techniques to reverse engineer the human brain have successfully inspired artificial intelligence algorithms. Recent advances in deep learning have achieved remarkable results in many domains such as visual object recognition, speech recognition and automated translation. However, although learning from sparse data is a common ability of the human mind, current machine learning techniques have not been able to mimic such ability with the same success.

\medskip

Previous computational cognitive research has proposed the idea that the ubiquitous ability of the human being to make predictions from sparse data can be represented by models of probabilistic inference over symbolically structured representation spaces. These proposals are revamping Jerry Fodor's hypothesis which states that thinking takes form in a sort of mental {\em Language of Thought} composed of a limited set of atomic symbols that can be combined to form more complex structures following combinatorial rules.

\medskip

In this work we design and evaluate different Language of Thought models to explain human learning from sparse data in various domains: binary sequences in the visual and auditory domain, geometric sequences in the visual domain, and logical concepts. In our models we assume that the Language of Thought acts as a programming language capable of generating programs to model concepts in the world, and we explain learning as a process of probabilistic inference over these programs or with a minimum length of description approach based on the notions of algorithmic complexity. Finally, we propose different techniques to improve the process of construction and validation of the Language of Thought models in order to make them more dynamic and reliable.
\vfill

\noindent\textbf{Keywords:} Language of Thought, Bayesian Inference, Kolmogorov Complexity, Minimum Description Length.

\newpage

