%!TEX root = ../main.tex


    \color{blue}
    Propuesta: acá poner la gramática y semántica FORMAL del lenguaje de geometría y su simplificación para cadenas binarias

    Todo lo que esté en blue en cap 1, 2, 3 iría en esta parte 1

    Eg:

    Definir  gramáticas   \gramgeo y    \grambin acá y también \mdlbin acá

    Para la semántica de los programas $P$ se puede también usar $\sem{P}$, como hacemos con las fórmulas, pero en este caso, la semántica de un programa es una secuencia en el octágono (o binaria). Habría que definir $\sem{P}$ relativo a un punto inicial $n\in\{0\dots 7\}$ de cómputo: 
    por ejemplo. algo como $\llbracket$\verb#+1#$\rrbracket_n= n+1 \mod 8$.% no me anda usar \verb adentro de \sem :( No deja usar \verb en modo math

    Antes lo llamamos $P(n)$, pero creo que es mejor llamarlo $\sem{P}_n$, asi $\sem{.}$ siempre quiere decir semántica (con fórmulas o secuencias)

    Podemos definir \mdlbinfrag adentro del cap 3 porque es muy particular

    usar este font \verb#[+2]^4# para programas. Ojo, no anda adentro de comando para semántica

    en cap 2 cambié ``la complejidad LoT", ``complejidad LoT", etc por  \mdlbin (es un comando) porque en el contexto de la tesis, quedaba demasiado general hablar de LoT, dado que tenemos varios LoTs

    También cambié  ``complejidad LoT de fragmentos" por  \mdlbinfrag (es un comando)




    \color{black}
