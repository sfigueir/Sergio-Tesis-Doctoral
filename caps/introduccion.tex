\chapter{Introducción}
\label{intro}
\section{Conectivismo y teoría computacional de la mente}
Explicar conectivismo y teoría computacional de la mente.
Debate similiradid/estadistica y simbolico
Aparición de Probabilistic Language of Thought
Idea a desarrollar (tomado de la wikipedia)

Mientras el conexionismo se hacía cada vez más popular en la década de 1980, hubo una reacción contraria por parte de algunos investigadores, incluyendo a Jerry Fodor, Steven Pinker y otros. Argumentaban que el conexionismo tal y como se estaba desarrollando corría el peligro de olvidar lo que ellos veían como los progresos realizados por el enfoque clásico de la inteligencia artificial en los campos de la ciencia cognitiva y la psicología. La inteligencia artificial convencional argumenta que la mente opera mediante la realización de operaciones simbólicas puramente formales, como una máquina de Turing. Algunos investigadores señalaron que la tendencia hacia el conexionismo era un error, ya que significaba una reversión hacia el asociacionismo y el abandono de la idea de un lenguaje del pensamiento.

La reciente popularidad de los sistemas dinámicos en la filosofía de la mente (debido a las obras de autores como Tim van Gelder) ha añadido una nueva perspectiva al debate, algunos autores argumentan ahora que cualquier división entre el conexionismo y la IA convencional queda mejor caracterizada como una división entre la IA convencional y los sistemas dinámicos.

\section{Lenguaje del pensamiento}
Profundizar Fodor

\subsection{Gramáticas}
Explicar gramáticas
Explicar diferencia entre sintaxis y semántica

\subsection{Composición}
\subsubsection{Longitud Mínima de Descripción}
mdl
complejidad de kolmogorov
\subsubsection{Ciencia Cognitiva Bayesiana}
rational analysis
