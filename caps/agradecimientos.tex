%\addcontentsline{toc}{chapter}{\textbf{Agradecimientos.}} 
\chapter*{Agradecimientos}
Toda tesis doctoral lleva conlleva un esfuerzo colectivo grande y sintetiza años de vida. Tal vez lo más bonito de este trabajo haya sido el momento de escribir estas líneas y mirar esos años con comprensión y cariño por las personas que me han acompañado.

A mi familia toda, quienes estaban cuando llegué a este mundo, quienes se fueron incorporando, y quienes se fueron yendo, porque uno tuvo el privilegio de crecer rodeado de los más diversos afectos y consejos. Nombrarlos a cada uno y cada una corre serios riesgos de algún olvido accidental, prefiero que se sientan reconocidxs todxs en este párrafo. Sí quisiera nombrar a las tres personas que aprendí a reconocer primero: mi papá, mi mamá y mi hermana. Porque ellxs tres saben todo el esfuerzo que hicimos para llegar juntos hasta acá, para armarnos y rearmarnos, para aprender a vivir juntos, para aprender a vivir separados, para cuidarnos, para conocernos, para perdonarnos y para reírnos. De los tres aprendí el valor de la constancia y la capacidad de querer y cuidar al otro aún en sus errores. 

A mis amigos de SPCT que me han soportado en el secundario y todo lo que vino después. Tal vez una de las tradiciones más lindas de nuestro pueblo sea esa capacidad de conformar grupos de amigos que logran persistir el paso del tiempo, de las ideologías, de los éxitos y de los fracasos. Con ellos fuimos aprendiendo de lealtades, de risas y de silencios.

A mis compañeros y compañeros de militancia, que me enseñaron que los pensamientos individuales sirven de poco, y que el verdadero desafío es pensar y construir sueños colectivos. De todos ellos y ellas, voy a nombrar a Néstor y Cristina porque sé que todxs lxs que fueron y son mis compañerxs nos sentimos representadxs en ellxs, y porque fueron los que recuperaron para mi generación el fuego de la política como herramienta de transformación. También ellos dos tienen la responsabilidad de haberme contagiado el cariño por la ciencia y sus investigadores e investigadoras. Jamás hubiera pensado en hacer y terminar este doctorado si no fuera por la importancia que Néstor y Cristina nos trasmitieron de tener un sistema científico soberano al servicio de nuestro pueblo y por la visión de ellos y el esfuerzo de millones de argentinos y argentinas por hacerlo crecer en los años que siguieron a la crisis del 2001.

A toda la comunidad del CONICET y del Departamento de Computación de Exactas de la UBA, pero en especial a Mariano y Santiago: mis dos directores de tesis que me invitaron a pensar juntos problemas desafiantes. A Mariano le agradezco su creatividad inagotable y su capacidad de mezclar trayectorias, preguntas y sueños para conformar en esos primeros años el Laboratorio de Neurociencia Integrativa que se convirtió en el espacio de pensamiento académico más interesante sobre neurociencia, psicología y el mundo en general que me tocó conocer. A Santiago le agradezco todo, cada minuto de estos años de doctorado y cada línea de este documento. Si Néstor y Cristina me crearon la admiración por los investigadores, Santiago la encarnó tal cual los idealicé: inteligente, comprometido, humilde, un gran compañero y maestro al mismo tiempo. A él le debo no haber dejado el doctorado persiguiendo otros caminos y haber entendido realmente las preguntas y respuestas del enorme trabajo de estos años.

Por último, a Alejandra, mi compañera de vida. En estos más de 15 años juntos aprendí todo de ella y pudimos sobrevivir a las alegrías y a las tristezas de un mundo complejo buscando juntos los rincones donde se escondieron los sueños. Ambos sabemos que esta tesis doctoral no fue ni lo más difícil, ni lo más bonito de estos años.