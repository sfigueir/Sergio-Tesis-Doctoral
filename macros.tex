%PAPER PLOS ONE
\newcommand{\gram}{\mathcal G}
\newcommand{\Prog}{{\rm Prog}}
\newcommand{\prog}{{\rm prog}}
\newcommand{\geom}{\mathcal Geo}
\newcommand{\figref}[1]{Fig~\ref{#1}}

% create "+" rule type for thick vertical lines
\newcolumntype{+}{!{\vrule width 2pt}}

% create \thickcline for thick horizontal lines of variable length
\newlength\savedwidth
\newcommand\thickcline[1]{%
  \noalign{\global\savedwidth\arrayrulewidth\global\arrayrulewidth 2pt}%
  \cline{#1}%
  \noalign{\vskip\arrayrulewidth}%
  \noalign{\global\arrayrulewidth\savedwidth}%
}

% \thickhline command for thick horizontal lines that span the table
\newcommand\thickhline{\noalign{\global\savedwidth\arrayrulewidth\global\arrayrulewidth 2pt}%
\hline
\noalign{\global\arrayrulewidth\savedwidth}}


%PAPER PRE
\newcommand{\start}{\textsc{start}}
\newcommand{\bool}{\textsc{bool}}
\newcommand{\atom}{\textsc{atom}}
\newcommand{\oxor}{\oplus}
\newcommand{\oand}{\wedge}
\newcommand{\oor}{\vee}
\newcommand{\grambool}{\ensuremath{{\sf P}}\xspace}
\newcommand{\gramboolxor}{\ensuremath{{\sf P}^\oxor}\xspace}
\newcommand{\controla}{$\con^1$\xspace}
\newcommand{\controlb}{$\con^2_c$\xspace}
\newcommand{\controlc}{$\con^3_c$\xspace}
\newcommand{\controld}{$\con^4_c$\xspace}
\newcommand{\targeta} {$\con^1$\xspace}
\newcommand{\targetb} {$\con^2_t$\xspace}
\newcommand{\targetc} {$\con^3_t$\xspace}
\newcommand{\targetd} {$\con^4_t$\xspace}
\newcommand{\testa}   {$\con^5$\xspace}
\newcommand{\testb}   {$\con^6$\xspace}

\newcommand{\sem}[1]{\llbracket #1\rrbracket}
\newcommand{\vars}{{\sf Vars}}
%\newcommand{\con}{{\mathcal C}}
%\newcommand{\mdl}[1]{\textsc{mdl}_{#1}}


%% BRM
\usepackage[textwidth=0.89in,textsize=scriptsize]{todonotes}
 \newcommand{\sergio}[1]{\todo[color=red!20]{{\bf Sergio:} #1}\xspace}
 \newcommand{\romano}[1]{\todo[color=blue!20]{{\bf Romano:} #1}\xspace}
 \newcommand{\pablo}[1]{\todo[color=orange!20]{{\bf Pablo:} #1}\xspace}
 \newcommand{\widesergio}[1]{\todo[inline,color=red!20]{{\bf Sergio:} #1}}

 \newcommand{\santi}[1]{\todo[color=green!20]{{\bf Santi:} #1}\xspace}
 \newcommand{\widesanti}[1]{\todo[inline,color=green!20]{{\bf Santi:} #1}}

%\newcommand{\marcaEnTabla}{{\bullet}}%\checkmark

\usepackage{setspace}
\doublespacing
\usepackage{breakcites}
\usepackage{lineno}


\usepackage{tabularx,multirow}
\usepackage[page]{appendix}
\usepackage{macros}

\usepackage{enumitem}% Permite poner cosas del tipo \begin{enumerate}[label=\Alph*]

\usepackage{fancyhdr}

\pagestyle{fancy}
%\fancyhf{}
\fancyhead{}
\renewcommand{\headrulewidth}{0pt} % para que no ponga la línea del header.
\fancyfoot{}
\rhead{\thepage} % esquina superior derecha



\usepackage{pgfgantt}
\usetikzlibrary{decorations.pathmorphing, shadows, shadings}

\hypersetup{
    colorlinks=true,   % color instead of boxes
    citecolor=blue,    % cite links in blue
    linkcolor=blue,    % other internal links in gray
%    linktocpage,         % in TOC, LOF and LOT, the link is on the page number
}


\theoremstyle{plain}
\theoremstyle{plain}
\newtheorem{theorem}{Theorem} %%% Original

\newtheorem{proposition}[theorem]{Proposition}
\newtheorem{lemma}[theorem]{Lemma}
\newtheorem{corollary}[theorem]{Corollary}
\newtheorem{remark}[theorem]{Remark}
\newtheorem{observation}[theorem]{Observation} 
\newtheorem{sketch}[theorem]{Sketch} 
\newtheorem{acknowledgements}[theorem]{Acknowledgements}
%
\newtheorem{fact}[theorem]{Fact}