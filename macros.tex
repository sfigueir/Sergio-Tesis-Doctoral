%PAPER PLOS ONE
\newcommand{\gram}{\mathcal G}
\newcommand{\Prog}{{\rm Prog}}
\newcommand{\prog}{{\rm prog}}
\newcommand{\geom}{\mathcal Geo}
\newcommand{\figref}[1]{Fig~\ref{#1}}

% create "+" rule type for thick vertical lines
\newcolumntype{+}{!{\vrule width 2pt}}

% create \thickcline for thick horizontal lines of variable length
\newlength\savedwidth
\newcommand\thickcline[1]{%
  \noalign{\global\savedwidth\arrayrulewidth\global\arrayrulewidth 2pt}%
  \cline{#1}%
  \noalign{\vskip\arrayrulewidth}%
  \noalign{\global\arrayrulewidth\savedwidth}%
}

% \thickhline command for thick horizontal lines that span the table
\newcommand\thickhline{\noalign{\global\savedwidth\arrayrulewidth\global\arrayrulewidth 2pt}%
\hline
\noalign{\global\arrayrulewidth\savedwidth}}


\newcommand{\NN}{\mathbb{N}}
\newcommand{\RR}{\mathbb{R}}
\newcommand{\concat}{{}^\smallfrown}
\newcommand{\last}{{\rm last}}
\newcommand\+[1]{\mathcal{#1}}
\newcommand{\refl}{{\rm Refl}}
\newcommand{\pref}{{\rm Pref}}
\newcommand{\eqdef}{\stackrel{\scriptscriptstyle \mathrm{def}}{=}}


%PAPER PRE
\newcommand{\start}{\textsc{start}}
\newcommand{\bool}{\textsc{bool}}
\newcommand{\atom}{\textsc{atom}}
\newcommand{\oxor}{\oplus}
\newcommand{\oand}{\wedge}
\newcommand{\oor}{\vee}
\newcommand{\grambool}{\ensuremath{{\sf P}}\xspace}
\newcommand{\gramboolxor}{\ensuremath{{\sf P}^\oxor}\xspace}
\newcommand{\controla}{$\con^1$\xspace}
\newcommand{\controlb}{$\con^2_c$\xspace}
\newcommand{\controlc}{$\con^3_c$\xspace}
\newcommand{\controld}{$\con^4_c$\xspace}
\newcommand{\targeta} {$\con^1$\xspace}
\newcommand{\targetb} {$\con^2_t$\xspace}
\newcommand{\targetc} {$\con^3_t$\xspace}
\newcommand{\targetd} {$\con^4_t$\xspace}
\newcommand{\testa}   {$\con^5$\xspace}
\newcommand{\testb}   {$\con^6$\xspace}

\newcommand{\sem}[1]{\llbracket #1\rrbracket}
\newcommand{\vars}{{\sf Vars}}
\newcommand{\con}{C}


\newcommand{\gramgeo}{{\sf geo}\xspace} %gramática para lenguaje de geometria
\newcommand{\grambin}{{\sf bin}\xspace} %gramática para lenguaje binario

\newcommand{\mdlname}{\textsc{mdl}}
\newcommand{\mdl}[1]{\ensuremath{\mdlname_{#1}}}

\newcommand{\mdlbin}{\ensuremath{\mdlname_{\grambin}}\xspace}
\newcommand{\mdlbinfrag}{\ensuremath{\mdlbin^{{\sf frag}}\xspace}}


%% BRM
%\newcommand{\marcaEnTabla}{{\bullet}}%\checkmark

% \usepackage{setspace}
% \doublespacing
% \usepackage{breakcites}
% \usepackage{lineno}


% \usepackage{tabularx,multirow}
% \usepackage[page]{appendix}
% \usepackage{macros}

\usepackage{enumitem}% Permite poner cosas del tipo \begin{enumerate}[label=\Alph*]

% \usepackage{fancyhdr}

% \pagestyle{fancy}
% %\fancyhf{}
% \fancyhead{}
% \renewcommand{\headrulewidth}{0pt} % para que no ponga la línea del header.
% \fancyfoot{}
% \rhead{\thepage} % esquina superior derecha



% \usepackage{pgfgantt}
% \usetikzlibrary{decorations.pathmorphing, shadows, shadings}

% \hypersetup{
%     colorlinks=true,   % color instead of boxes
%     citecolor=blue,    % cite links in blue
%     linkcolor=blue,    % other internal links in gray
% %    linktocpage,         % in TOC, LOF and LOT, the link is on the page number
% }


\theoremstyle{plain}
\theoremstyle{plain}
\newtheorem{theorem}{Theorem} %%% Original

\newtheorem{proposition}[theorem]{Proposition}
\newtheorem{lemma}[theorem]{Lemma}
\newtheorem{corollary}[theorem]{Corollary}
\newtheorem{remark}[theorem]{Remark}
\newtheorem{observation}[theorem]{Observation} 
\newtheorem{sketch}[theorem]{Sketch} 
\newtheorem{acknowledgements}[theorem]{Acknowledgements}
%
\newtheorem{fact}[theorem]{Fact}

\theoremstyle{definition}
\newtheorem{definicion}[theorem]{Definición}
\newtheorem{teorema}[theorem]{Teorema}
\newtheorem{corolario}[theorem]{Corolario}
\newtheorem{proposicion}[theorem]{Proposición}
\newtheorem{lema}[theorem]{Lema}
\newtheorem{hecho}[theorem]{Hecho}



\newcommand{\varA}{p_1}
\newcommand{\varB}{p_2}
\newcommand{\varC}{p_3}
\newcommand{\varD}{p_4}
\newcommand{\varE}{p_5}
\newcommand{\varF}{p_6}
\newcommand{\varG}{p_7}
\newcommand{\varH}{p_8}

\newcommand{\AND}{\ensuremath{\land}\xspace}
\newcommand{\OR}{\ensuremath{\lor}\xspace}

\newcommand{\variables}[1]{\ensuremath{\textsc{var}(#1)}}% Para mencionar variables de una f\'ormula
\newcommand{\propvars}{\textsc{Prop}}



% DE LA TESIS DE SANTIAGO (EN INTRO)
% NAMES
\newcommand{\kolcomp}{Kolmogorov complexity\xspace}
\newcommand{\Kolcomp}{Kolmogorov complexity\xspace}
\newcommand{\pfree}{prefix-free\xspace}
\newcommand{\Pfree}{Prefix-free\xspace}


% SETS
\newcommand{\nat}{\mathbb{N}}
\newcommand{\natplus}{\mathds{N}^+}
\newcommand{\rea}{\mathds{R}}
\newcommand{\rat}{\mathds{Q}}
%\newcommand{\voc}{\{0,1\}}
\newcommand{\voc}{2}
\newcommand{\words}{\voc^{<\omega}}
\newcommand{\wordsn}[1]{\voc^{#1}}
\newcommand{\wordsupton}[1]{\voc^{\leq{#1}}}
\newcommand{\cantor}{\voc^{\omega}}
\newcommand{\mix}{\voc^{\leq\omega}}
\newcommand{\emptystring}{\lambda}

% STRINGS AND PROGRAMS
\newcommand{\sta}{\sigma}
\newcommand{\stb}{\tau}
\newcommand{\stc}{\chi}
\newcommand{\pra}{\rho}
\newcommand{\prb}{\gamma}
\newcommand{\prc}{\nu}

% KOLMOGOROV COMPLEXITIES
\newcommand{\K}{K} %Prefix Kolmogorov complexity
\newcommand{\C}{C} %Plain Kolmogorov complexity
\newcommand{\Ktime}[1]{\K_{#1}} %Prefix Kolmogorov complexity at step
\newcommand{\CU}{\C_{\U}} %Plain Kolmogorov complexity based on U
\newcommand{\CM}{\C_{\M}} %Plain Kolmogorov complexity based on M
\newcommand{\Ctime}[1]{C_{#1}} %Plain Kolmogorov complexity at step
\newcommand{\KU}{\K_{\U}} %Prefix Kolmogorov complexity based on U
\newcommand{\KV}{\K_{\V}} %Prefix Kolmogorov complexity based on V
\newcommand{\KM}{\K_{\M}} %Prefix Kolmogorov complexity based on U

% MACHINES
\newcommand{\U}{{\bf U}}  % universal machine U
\newcommand{\Us}{\U_s}  % machine U at step s
\newcommand{\Ut}{\U_t}  % machine U at step t
\newcommand{\Uu}{\U_u}  % machine U at step u
\newcommand{\Utime}[1]{\U_{#1}}  % machine U at certain time
\newcommand{\V}{{\bf V}}  % universal machine V
\newcommand{\Vs}{\V_s}  % machine V at step s
\newcommand{\W}{{\bf W}}  % universal machine W
\newcommand{\M}{{\bf M}}  % particular machine M
\newcommand{\N}{{\bf N}}  % particular machine M
\newcommand{\Ms}{{\bf M}_s}  % particular machine M at step s
\newcommand{\Mt}{{\bf M}_t}  % particular machine M at step t
\newcommand{\Mu}{{\bf M}_u}  % particular machine M at step u
\newcommand{\T}{{\bf T}}  % enumeration of Turing machines
\newcommand{\Uinf}{{\U}^\infty}
\newcommand{\Uinft}{{\U}^\infty_t}

\newcommand{\size}[1]{\| #1 \|} % number of elements of a set
\newcommand{\len}[1]{| #1 |} % length of a string
\newcommand{\rec}{computable\xspace}
\newcommand{\recly}{computably\xspace}
\newcommand{\comp}{computable\xspace}
\newcommand{\comply}{computably\xspace}
\newcommand{\ce}{c.e.\xspace}

\newcommand{\opt}{universal\xspace}
\newcommand{\Opt}{Universal\xspace}
\newcommand{\optity}{universality\xspace}
\newcommand{\Optity}{Universality\xspace}
\newcommand{\abs}[1]{\left| #1 \right|} % absolute value
%\newcommand{\abs}[1]{| #1 |} % absolute value
\newcommand{\then}{\Rightarrow}
\newcommand{\pair}[2]{\langle #1 , #2 \rangle}
\newcommand{\wt}[1]{{\rm wt}\left(#1\right)} % weight of a set (for example, a Kraft-Chaitin set
\newcommand{\dup}[1]{\overline{#1}} % interleave 0s except in the last position.
\newcommand{\measure}[1]{\mu\left(#1\right)}
\newcommand{\open}[1]{ #1\cantor } %open set
\newcommand{\domain}{\qopname\relax{no}{dom}}
