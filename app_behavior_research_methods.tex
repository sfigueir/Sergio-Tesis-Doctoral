%!TEX root = main.tex

\chapter{Apéndice del Capítulo~\ref{chapter:BRM}}

% \section{Exclusion criteria and data processing}\label{Sec:ExclusionCriteria}
\section{Criterios de exclusión y tratamiento de datos}\label{Sec:ExclusionCriteria}

% We decided to collect data for up to 3 weeks or until we reached a total of 100 participants. Via restrictions on the platform where the experiment was conducted, participants that took more than 4 hours or who did not complete all the trials were automatically excluded from the analysis. We were also prepared to exclude afterward the results from those participants whose verbal explanations denoted the use of external aids or methods outside the scope of the paper, such as using external help or taking screenshots of the concept, but there were no clear-cut cases of that behaviour ($N=0$).  
Decidimos recopilar datos durante un máximo de 3 semanas o hasta llegar a un total de 100 participantes. A través de restricciones en la plataforma donde se realizó el experimento, los participantes que tardaron más de 4 horas o que no completaron todos las pruebas fueron automáticamente excluidos del análisis. Estábamos preparados para excluir posteriormente los resultados de aquellos participantes cuyas explicaciones verbales denotaban el uso de ayudas externas o métodos fuera del alcance del artículo, como la toma de capturas de pantalla del concepto. Sin embargo, no hubo casos claros de ese comportamiento ($ N = 0 $).

% Additionally, while our preregistered exclusion criteria did not encompass the potential cases of written explanations that were legitimate but indicative of use of rules extraneous to propositional logic or to our semantic framework, in the end we did not detect any of these cases. This encouraging result is weakly indicative of the usefulness of our careful considerations for building adequate semantic representations, as mentioned in Section~\ref{Experiment_design}. For the comprehensive written explanations of the participants, we refer the reader to the uploaded raw data at \url{https://osf.io/gtuwp/}.  
Además, si bien nuestros criterios de exclusión prerregistrados no abarcaron los casos potenciales de explicaciones escritas que eran legítimas pero indicativas del uso de reglas ajenas a la lógica proposicional o a nuestro marco semántico, al final no detectamos ninguno de estos casos. Este resultado alentador es débilmente indicativo de la utilidad de nuestras cuidadosas consideraciones para construir representaciones semánticas adecuadas, como se menciona en la Sección~\ref{Experiment_design}. Para las explicaciones escritas completas de los participantes, remitimos al lector a los datos crudos cargados en \url{https://osf.io/gtuwp/}.

% Balanced division into the two groups was handled via the psiTurk library, which decides the group a new worker will be assigned to, based on the current number of completed experiments in each group. 
La división equilibrada en los dos grupos se manejó a través de la biblioteca psiTurk, que decide el grupo al que se asignará un nuevo trabajador, en función del número actual de experimentos completados en cada grupo.

% We ignored individual trails from participants that in the generalization stage chose a generalization inconsistent with any valid explanation (but this did not provoke the exclusion of other independent trials by the same participant). See Section~\ref{Results} for details. 
Ignoramos los rastros individuales de los participantes que en la etapa de generalización eligieron una generalización inconsistente con cualquier explicación válida (pero esto no provocó la exclusión de otras pruebas independientes por parte del mismo participante). Consultar la Sección~\ref{Results} para obtener más detalles.

% \section{Pilot} \label{subsection:resultsPilot}
\section{Prueba piloto} \label{subsection:resultsPilot}

% This experiment  is  informed by a previous pilot with 22 participants, which we executed in order to have some validation for our expected effects before making the preregistration. This pilot used more complex pairs of concepts, with a longer minimum description length for the two corresponding rules, and where using both $\AND$ and $\OR$ in the same rule was often necessary.  Originally, we expected a naturally arising separation into different groups, depending on the features of explanation found for the first trial. However, we encountered a very strong preference for explanations using solely $\AND$, and this prompted various changes in the final design of the experiment that was preregistered in the OSF version.
Este experimento fue encarado por un prueba piloto con 22 participantes, que ejecutamos con el fin de tener alguna validación de nuestros efectos esperados antes de realizar el pre-registro. Esta prueba piloto utilizó pares de conceptos más complejos, con una longitud mínima de descripción más larga para las dos reglas correspondientes, y donde a menudo era necesario usar tanto $\AND $ como $\OR $ en la misma regla. Originalmente, esperábamos una separación que surgiera naturalmente en diferentes grupos, dependiendo de las características de explicación encontradas para la primera prueba. Sin embargo, encontramos una preferencia muy fuerte por las explicaciones que usan únicamente $ \AND $, y esto provocó varios cambios en el diseño final del experimento que fue prer-registrado en la versión OSF.

% More precisely, in our first trial in that pilot, 81\% ($N = 18$) of the workers explained the (incomplete) concept as a conjunction of three variables, while only 9\% ($N = 2$) explained it as a disjunction of two. This happened even though we had made the $\AND$ explanation longer with the intention to compensate for the relative ease of $\AND$ with respect to $\OR$ (so as to avoid getting a statistically inadequate number of participants self-selecting to the $\OR$ case). This result goes in line with known work about the relative hardness of learning concepts with the $\OR$ operator~\cite{bourne1970knowing}. In our framework of more than one plausible rule, a possible explanation to this population disparity could be that, when looking for common characteristics, it is natural to search first for individual features that always appear. Another explanation could be that, in a universe with low number of features, repetition of many of them becomes very salient, and thus the relation between hardness and number of conjunctions is not necessarily monotonic. In any case, this result was not part of the preregistration, so it is presented here only as an indication of an interesting effect to study.
Más precisamente, en nuestra primera prueba en ese piloto, el 81\% ($ N = 18 $) de los trabajadores explicó el concepto (incompleto) como una conjunción de tres variables, mientras que solo el 9\% ($ N = 2 $) lo explicó como una disyunción de dos. Esto sucedió a pesar de que habíamos alargado la explicación de $ \AND $ con la intención de compensar la relativa facilidad de $ \AND $ con respecto a $ \OR $ (para evitar que un número estadísticamente inadecuado de participantes se autoseleccionados al caso $ \OR $). Este resultado concuerda con el trabajo conocido sobre la relativa dificultad de aprender conceptos con el operador $ \OR $~\cite{bourne1970knowing}. En nuestro marco de más de una regla plausible, una posible explicación a esta disparidad poblacional podría ser que, al buscar características comunes, es natural buscar primero las características individuales que siempre aparecen. Otra explicación podría ser que, en un universo con un número reducido de características, la repetición de muchas de ellas se vuelve muy destacada y, por tanto, la relación entre dificultad y número de conjunciones no es necesariamente monótona. En cualquier caso, este resultado no formó parte del pre-registro, por lo que se presenta aquí solo como una indicación de un efecto interesante a estudiar.



% \section{Technical results}\label{Sec:MainTheoremConcept}
\section{Resultados técnicos}\label{Sec:MainTheoremConcept}
% Let us fix a non-empty set of propositional variables \propvars. A valuation is formally defined as a function ${v:\propvars\rightarrow\{0,1\}}$ that determines the truth value of the propositional variables. A valuation can be extended in the standard way to preserve the usual semantics of Boolean operators and thus to determine the truth value of propositional formulas (which we call `rules' in the context of describing concepts).  We say that a valuation $v$ satisfies a formula~$\varphi$ if $v(\varphi)=1$. We say that a formula $\varphi$ is a contingency if there exist a valuation $v_t$ that satisfies it and a valuation $v_f$ that does not.
Fijemos un conjunto no vacío de variables proposicionales \propvars. Una valuación se define formalmente como una función $ {v: \propvars \rightarrow \{0,1 \}} $ que determina el valor de verdad de las variables proposicionales. Una valuación puede extenderse de la forma estándar para preservar la semántica habitual de los operadores booleanos y así determinar el valor de verdad de las fórmulas proposicionales (que llamamos `reglas' en el contexto de la descripción de conceptos). Decimos que una valuación $ v $ satisface una fórmula~$\varphi $ si $ v(\varphi) = 1 $. Decimos que una fórmula $ \varphi $ es una contingencia si existe una valuación $ v_t $ que la satisface y una valuación $ v_f $ que no.

% Given a propositional formula $\varphi$, we define $\variables{\varphi}$ as the set of variables that appear in it. For example, if $\varphi_{e} = p_1 \lor (p_2 \land \lnot p_2)$, then $\variables{\varphi_e} = \{p_1, p_2\}$.
Dada una fórmula proposicional $\varphi$, definimos $\variables{\varphi}$ como el conjunto de variables que aparecen en $\varphi$. Por ejemplo, si $\varphi_{e} = p_1 \lor (p_2 \land \lnot p_2)$, entonces $\variables{\varphi_e} = \{p_1, p_2\}$.

% We say that a formula $\varphi$ is variable-minimal if there is no other formula $\psi$ such that the truth values of $\varphi$ and $\psi$ coincide over all valuations and $\variables{\psi} \subsetneq \variables{\varphi}$. For example, the previous $\varphi_e$ is not variable-minimal, since it is equivalent to $\psi = p_1$, which uses one less propositional variable. 
Decimos que una fórmula $\varphi$ es variable-minimal si no existe otra fórmula $\psi$ tal que los valores de verdad de $\varphi$ y $\psi$ coincidan sobre todas las valuaciones y $\variables{\psi} \subsetneq \variables{\varphi}$. Por ejemplos, la $\varphi_e$ previa no es variable-minimal, dado que es equivalente a $\psi = p_1$, que usa una variable proposicional menos. 

% We begin by proving a very basic lemma for illustrative purposes. 
Comenzamos probando un lema muy básico con fines ilustrativos.
\begin{lemma}\label{lemma:InterseccionUnionConceptos}
% Let $\varphi_1$ and $\varphi_2$ be two contingencies such that $\variables{\varphi_1} \cap  \variables{\varphi_2} = \emptyset$. Then there exists a valuation $v_{in}$ such that $v_{in}$ satisfies both $\varphi_1$ and $\varphi_2$, and a valuation $v_{out}$ that satisfies neither $\varphi_1$ nor $\varphi_2$.
Sean $\varphi_1$ y $\varphi_2$ dos contingencias tales que $\variables{\varphi_1} \cap  \variables{\varphi_2} = \emptyset$. Entonces existe una valuación $v_{in}$ tl que $v_{in}$ satisface $\varphi_1$ and $\varphi_2$, y una valuación $v_{out}$ no satisface $\varphi_1$ ni $\varphi_2$.
\end{lemma}
% In other words, the lemma says that when we have two non-trivial concepts concerning non-overlapping sets of features, then there is at least one (positive) example that satisfies both concepts simultaneously and at least one (negative) example that satisfies none of them. 
En otras palabras, el lema dice que cuando tenemos dos conceptos no triviales sobre conjuntos de características que no se superponen, entonces hay al menos un ejemplo (positivo) que satisface ambos conceptos simultáneamente y al menos un ejemplo (negativo) que no satisface ninguno. de ellos.

\begin{proof}
% Whether a valuation satisfies or not a formula $\varphi$ depends only on how it evaluates propositional variables on $\variables{\varphi}$. Since $\variables{\varphi_1} \cap  \variables{\varphi_2} = \emptyset$ and both formula are satisfiable via some $v_1$ and $v_2$ respectively, we can construct a valuation $v_{in}$ by joining the values of $v_1, v_2$ on the (disjoint) sets of variables of each formula: $v_{in}(p) = v_1(p)$ if $p \in \variables{\varphi_1}$,  $v_{in}(p) = v_2(p)$ if $p \in \variables{\varphi_2}$, and $v_{in}(p) = 0$ otherwise. 
El hecho de que una valuación satisfaga o no una fórmula $\varphi$ depende solo de cómo se evalúan las variables proposicionales en $\variables{\varphi}$. Como $\variables{\varphi_1} \cap  \variables{\varphi_2} = \emptyset$ y ambas fórmulas son satisfacibles vía algunas $v_1$ y $v_2$ respectivamente, podemos construir valuaciones $v_{in}$ juntando los valores de $v_1, v_2$ en los conjuntos (disjuntos) de variables de cada fórmula: $v_{in}(p) = v_1(p)$ si $p \in \variables{\varphi_1}$,  $v_{in}(p) = v_2(p)$ si $p \in \variables{\varphi_2}$, y $v_{in}(p) = 0$ en caso contrario. 

% Similarly, since $\varphi_1, \varphi_2$ are not contingencies, there exist valuations $\bar{v}_1$ and $\bar{v}_2$ that do not satisfy $\varphi_1$ and $\varphi_2$ respectively. We use these valuations as before to construct a valuation $v_{out}$ that does not satisfy $\varphi_1$ nor $\varphi_2$, as we wanted.
Análogamente, como $\varphi_1, \varphi_2$  no son contingencias, existen valuaciones $\bar{v}_1$ y $\bar{v}_2$ que no satisfacen $\varphi_1$ ni $\varphi_2$ respectivamente. Usamos estas valuaciones como antes para construir valuaciones $v_{out}$ que no satisfacen $\varphi_1$ ni $\varphi_2$, como queríamos.
\end{proof}


\begin{lemma} \label{lemma:variableMinimalProp}
% If $\varphi$ is a variable-minimal contingency, and $p \in \variables{\varphi}$, then there exists a valuation $v$ such that $v$ satisfies $\varphi$ but $\tilde{v}$ does not, where $\tilde{v}$ is the single valuation that coincides with $v$ except on $p$.
Si $\varphi$ es una contingencia variable-minimal, y $p \in \variables{\varphi}$, entonces existe una valuación $v$ tal que $v$ satisface $\varphi$ pero $\tilde{v}$ no, donde $\tilde{v}$ es la única valuación que coincide con $v$ excepto en $p$.
\end{lemma}
\begin{proof}

% By way of contradiction, assume the conclusion does not hold: that for any valuation, its satisfaction of $\varphi$ is independent of its value on $p$. In this case, necessarily $\{p\} \neq \variables{\varphi}$, or otherwise $\varphi$ would not be a contingency (as it would always be true or always false).
Por el absurdo, supongamos que la conclusión no es válida: que para cualquier valuación, la verdad de $ \varphi $ es independiente de su valor en $ p $. En este caso, necesariamente $ \{p \} \neq \variables {\varphi} $, o de lo contrario $ \varphi $ no sería una contingencia (ya que siempre sería verdadera o siempre falsa).

% Now consider $V_\varphi$ the (non-empty) set of valuations that satisfy $\varphi$, and consider $V^{-p}_\varphi$ its restriction to $\variables{\varphi} \backslash\{p\}$. From $V^{-p}_\varphi$ we can construct, in a standard way via truth tables, a formula $\tilde{\varphi}$ with $\variables{\tilde{\varphi}} = \variables{\varphi} \backslash\{p\}$ such that a valuation $v$ satisfies $\tilde{\varphi}$ if and only if $v|_{ \variables{\tilde{\varphi}}} \in V^{-p}_\varphi$. Since by assumption the value of $p$ does not matter for $\varphi$, we have by construction that $\varphi$ is equivalent to $\tilde{\varphi}$, but $\variables{\tilde{\varphi}} \subsetneq \variables{\varphi}$, which contradicts the variable-minimality of $\varphi$.
Sea $V_\varphi$ el conjunto (no vacío) de valuaciones que satisface $\varphi$, y sea $V^{-p}_\varphi$ su restricción a $\variables{\varphi} \backslash\{p\}$. A partir de $V^{-p}_\varphi$ podemos construir de forma estándar a través de tablas de verdad, una fórmula $\tilde{\varphi}$ con $\variables{\tilde{\varphi}} = \variables{\varphi} \backslash\{p\}$ tal que una valuación $v$ satisfaga $\tilde{\varphi}$ si y solo si $v|_{ \variables{\tilde{\varphi}}} \in V^{-p}_\varphi$. Como la suposición que el valor de $p$ no importa para $\varphi$, tenemos por construcción que $\varphi$ es equivalente a $\tilde{\varphi}$, pero $\variables{\tilde{\varphi}} \subsetneq \variables{\varphi}$, que contradice la variable-minimalidad de $\varphi$.
\end{proof}


% The following theorem shows the general theoretical correctness of our experimental setup. It says that if we show as positive examples the full intersection of two non-trivial concepts whose minimal descriptions contain no features in common, and show as negative examples the complement of the union of both concepts, any rule used to explain the seen (incomplete) concept must use a superset of the variables used to minimally describe one of these concepts. Otherwise, the chosen rule would be incompatible with the known data.  
El siguiente teorema muestra la corrección teórica general de nuestra configuración experimental. Dice que si mostramos como ejemplos positivos la intersección completa de dos conceptos no triviales cuyas descripciones mínimas no contienen características en común, y mostramos como ejemplos negativos el complemento de la unión de ambos conceptos, cualquier regla utilizada para explicar el concepto visto (incompleto) debe utilizar un superconjunto de las variables utilizadas para describir mínimamente uno de estos conceptos. De lo contrario, la regla elegida sería incompatible con los datos conocidos.

\begin{theorem}\label{theorem:TeoremaPrincipal}
% Let $\varphi_1$ and $\varphi_2$ be two variable-minimal contingencies such that $\variables{\varphi_1}\cap \variables{\varphi_2} = \emptyset$. Let $\psi$ be a formula such that $\variables{\psi} \cap \variables{\varphi_1} \neq \variables{\varphi_1}$ and such that $\variables{\psi} \cap \variables{\varphi_2} \neq \variables{\varphi_2}$.  Furthermore, assume that for all valuations $v$ that satisfy $\varphi_1 \land \varphi_2$, $v$ also satisfies $\psi$. Then there exist two valuations $v_{in}, v_{out}$ such that:
Sean $\varphi_1$ y $\varphi_2$ dos contingencias variable-minimales tales que $\variables{\varphi_1}\cap \variables{\varphi_2} = \emptyset$. Sea $\psi$ una fórmula tal que $\variables{\psi} \cap \variables{\varphi_1} \neq \variables{\varphi_1}$ y tal que $\variables{\psi} \cap \variables{\varphi_2} \neq \variables{\varphi_2}$.  Furthermore, assume that for all valuations $v$ that satisfy $\varphi_1 \land \varphi_2$, $v$ also satisfies $\psi$. Then there exist two valuations $v_{in}, v_{out}$ such that:
\begin{enumerate}
    % \item \label{item:v1EnInterseccion} $v_{in}$ satisfies $\varphi_1 \land \varphi_2$
    \item \label{item:v1EnInterseccion} $v_{in}$ satisface $\varphi_1 \land \varphi_2$:    
    % \item $v_{out}$ does not satisfy $\varphi_1 \lor \varphi_2$ 
    \item $v_{out}$ no satisface $\varphi_1 \lor \varphi_2$; 
    
    % \item $v_{in}$ and $v_{out}$ both satisfy $\psi$.
    \item $v_{in}$ y $v_{out}$ satisfacen $\psi$.
\end{enumerate}
\end{theorem}
\begin{proof}
% From the hypotheses we know that there is a variable $p_1 \in \variables{\varphi_1} \backslash \variables{\psi}$ and a variable $p_2 \in \variables{\varphi_2} \backslash \variables{\psi}$.  Since $\varphi_1, \varphi_2$ are variable-minimal contingencies, from Lemma~\ref{lemma:variableMinimalProp} we have that there exist valuations $v_1$ and $v_2$ such that they satisfy $\varphi_1$ and $\varphi_2$ respectively, but where $\tilde{v}_1$ and $\tilde{v}_2$ do not, with $\tilde{v}_1$ and $\tilde{v}_2$ being the valuations that coincide with $v_1$ and $v_2$ save on $p_1$ and $p_2$ respectively. Using that $\variables{\varphi_1} \cap \variables{\varphi_2} = \emptyset$, we can construct from $v_1$ and $v_2$ (as we did in the proof of Lemma~\ref{lemma:InterseccionUnionConceptos}) a valuation $v_{in}$ such that $v_{in}$ satifies both $\varphi_1$ and $\varphi_2$, and also such that $v_{out}$ does not satisfy neither of them, where we take $v_{out}$ to coincide with $v_{in}$ save on $p_1$ and on $p_2$. From the hypothesis, necessarily $v_{in}$ satisfies $\psi$. However, since $\{p_1, p_2\} \cap \variables{\psi} = \emptyset$, the value over $p_1$ or $p_2$ does not matter for the satisfaction of $\psi$, and thus $v_{out}$ also satisfies $\psi$, as we wanted to see.
De las hipótesis sabemos que hay una variable $p_1 \in \variables{\varphi_1} \backslash \variables{\psi}$ y una variable $p_2 \in \variables{\varphi_2} \backslash \variables{\psi}$.  Como $\varphi_1, \varphi_2$ hay contingencias variable-minimales, del Lema~\ref{lemma:variableMinimalProp} tenemos que existen valuaciones $v_1$ y $v_2$ tales que satisfacen $\varphi_1$ y $\varphi_2$ respectivamente, pero donde $\tilde{v}_1$ y $\tilde{v}_2$ no lo hacen, donde $\tilde{v}_1$ y $\tilde{v}_2$ son las valuaciones que coinciden con $v_1$ y $v_2$ excepto en $p_1$ y $p_2$ respectivamente. Usando que $\variables{\varphi_1} \cap \variables{\varphi_2} = \emptyset$, podemos construir a partir de $v_1$ y $v_2$ (como lo hicimos en la prueba del Lema~\ref{lemma:InterseccionUnionConceptos}) una valuación $v_{in}$ tal que $v_{in}$ satisface  tanto $\varphi_1$ como $\varphi_2$, y también tal que $v_{out}$ no satisface ninguna de ellas, donde tomamos $v_{out}$ coincidente con $v_{in}$ salvo en $p_1$ y en $p_2$. Por hipótesis, necesariamente $v_{in}$ satisface $\psi$. Sin embargo, como $\{p_1, p_2\} \cap \variables{\psi} = \emptyset$, el valor sobre $p_1$ o $p_2$ es irrelevante para  $\psi$, y por lo tanto $v_{out}$ también satisface $\psi$, como queríamos probar.
\end{proof}

% Note that the statement of Theorem~\ref{theorem:TeoremaPrincipal} can be generalized to any number of non-trivial rules $\varphi_1, \dots, \varphi_n$ such that $\variables{\varphi_i} \cap \variables{\varphi_j} = \emptyset$ for all $i \neq j$, and with $\psi$ such that $\variables{\psi} \cap \variables{\varphi_i} \neq \variables{\varphi_i}$ for all $i$. This means that we can test concept learning under any multiplicity of possible explanations, as long as the underlying propositional universe is large enough and the rules are chosen adequately. 
Notar que el enunciado del Teorema~\ref{theorem:TeoremaPrincipal} se puede generalizar a cualquier número de reglas no triviales $\varphi_1, \dots, \varphi_n$ tales que $\variables{\varphi_i} \cap \variables{\varphi_j} = \emptyset$ para todo $i \neq j$, y donde $\psi$ satisfaga $\variables{\psi} \cap \variables{\varphi_i} \neq \variables{\varphi_i}$ para todo $i$. Esto significa que podemos testear el aprendizaje de conceptos bajo cualquier multiplicidad de explicaciones posibles, siempre que el universo proposicional subyacente sea lo suficientemente grande y las reglas se elijan adecuadamente.


